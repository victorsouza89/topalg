Introduziu-se nas seções anteriores alguns aspectos básicos da \textbf{teoria geométrica dos grupos}, uma área da matemática que se dedica ao estudo de grupos finitamente gerados por meio da exploração das conexões entre suas propriedades algébricas e as propriedades topológicas e geométricas de espaços relacionados.

A teoria geométrica dos grupos está diretamente relacionada à \textbf{topologia algébrica}, área da matemáticas que usa teorias geométricas e topológicas para estudar estruturas algébricas. No caso desses escritos, relacionamos os grupos a classes de espaços métricos e introduzimos uma propriedade interessante, o número de pontas de um espaço.

O desenvolvimento aqui realizado, embora simples e preliminar, permite a compreensão de fatos não tão triviais e tem algumas aplicações práticas imediatas. Por exemplo, tendo em vista os conhecimentos apresentados, fica clara a inexistência de isomorfismo entre $\mathbb{Z}$ e $\mathbb{Z}^n$ para $n \geq 2$, já que os espaços tem diferentes números de pontas (e se fossem isomorfos, seus grafos de Cayley também o seriam, implicando no mesmo número de pontas).

O desenvolvimento aqui apresentado baseou-se principalmente em \cite{posCur} e \cite{notasGGT}. Além disso, recomenda-se fortemente a leitura de \cite{hatcher}, que introduz os principais aspectos gerais de topologia algébrica.